\documentclass[a4paper,12pt,titlepage]{article}

\usepackage[left=2.5cm,right=1cm,top=2cm,bottom=2cm]{geometry}
% \documentclass{article}
\usepackage[T2A]{fontenc}
\usepackage[utf8]{inputenc}
\usepackage[english,russian,ukrainian]{babel}
\usepackage{indentfirst}
\usepackage{amssymb,amsmath}
\usepackage{fancyhdr}
\usepackage{graphicx}
\usepackage{amsmath, amsthm, amssymb}
\usepackage{mathtools}
\usepackage{tikz}
\usepackage{booktabs}
\usepackage{multirow}
\usepackage{mathrsfs}
\usepackage{pgfplots,pgfplotstable}
\pgfplotsset{compat=1.7}

\usepackage{comment}
\usepackage{array}
\usepackage{color}

\usepackage[ampersand]{easylist}
\ListProperties(Hide=100, Hang=true, Progressive=3ex, Style*=\tiny$\blacksquare$  , Style2*=$\bullet$, Style3*=$\circ$ )

\usepackage[hidelinks]{hyperref}
\usepackage[shortlabels]{enumitem}
\renewcommand{\baselinestretch}{1.5}

\newtheorem{definition}{Означення}[section]
\newtheorem{theorem}{Теорема}[section]
\newtheorem{lemma}{Лема}[section]
\newtheorem{proposition}{Твердження}[section]
\newtheorem{example}{Приклад}[section]
\newcommand{\N}{\mathbb{N}}
\newcommand{\Z}{\mathbb{Z}}
\newcommand{\Q}{\mathbb{Q}}
\newcommand{\R}{\mathbb{R}}
\addto\captionsukrainian{% Replace "english" with the language you use
  \renewcommand{\contentsname}%
    {\centering ЗМІСТ}%
  \renewcommand\refname{\centering СПИСОК ВИКОРИСТАНИХ ДЖЕРЕЛ}
}

\begin{document}
\setcounter{page}{2}
\pagestyle{myheadings}
\renewcommand{\headrulewidth}{0 pt}
\begin{titlepage}
\begin{center}

ЛЬВІВСЬКИЙ НАЦІОНАЛЬНИЙ УНІВЕРСИТЕТ ІМЕНІ ІВАНА ФРАНКА\\
Факультет прикладної математики та інформатики\\
Кафедра  математичного моделювання соціально-економічних процесів
\end{center}


\vspace{10em}

\begin{center}
\begin{center}
\textbf{ \Large Kурсова робота}
\end{center}

\textbf{"Техніка навчання на малому наборі даних для задача сегментації"}
\end{center}
\vspace{14em}
\begin{flushright}
Виконав: \\

студент V курсу, групи ПМІ-53 \\
напряму підготовки (спеціальності)\\ 
6.040302 -- "інформатика" \\
\textbf {Баранов М. В.} \\
{Керівник: Щербина Юрій Миколайович\\}
{доцент} \textbf{....} \\
{Рецензент:\\}
\underline{\;\;\;\;\;\;\;\;\;\;\;\;\;\;\;\;\;\;\;\;\;\;\;\;\;\;\;\;\;\;\;\;\;\;\;\;\;\;\;\;\;\;\;\;}
\end{flushright}

\vspace{\fill}

\begin{center}
2019
\end{center}

\end{titlepage}
\newpage
\tableofcontents
\newpage
	

        
\newpage
\section*{\centering РЕФЕРАТ}
\addcontentsline{toc}{section}{Реферат}
...

\newpage
\section*{\centering ВСТУП}	
\addcontentsline{toc}{section}{Вступ}	



Людський зір – один з найпотужніших органів чуттів. Існують різні оцінки кількості інформації, що надходить через органи зору. В середньому більше 80\%-90\% інформації людина отримує шляхом перетворення енергії електромагнітного випромінювання світлового діапазону та обробки цих сигналів мозком. За час еволюції зоровий апарт людини досягнув небачених висот: ми здатні розрізняти великі, дрібні предмети, відстані, положення, взаємне розташування; здатні знаходити подібності та відмінності між предметами, проводити узагальнення лише по одному екземпляру. Варто також відмітити швидкість роботи людського зору: мозок здатен обробляти 11 мільйонів біт сенсорної інформації в секунду, 10 мільйонів з яких припадає на зоровий апарт. Насправді, це не є захмарним числом. Якщо порівняти з кількістю інформації типового сучасного зображення формату FullHD — це лише одне зображення на хвилину! Проте людське око здатне опрацьовувати до 100 “зображень” в секунду, розрізняючи дрібніші деталі, які непомітні на цифрових фотографіях. На сьогодні досить детально досліджено процес роботи людського ока, але процес опрацювання отриманої досі приховує багато таємниць. В еру науково-технічної революції все більше попиту отримує завдання автоматичного аналізу зображень. Прикладів застосування машинного опрацювання візуальної інформації безліч: аналіз камер спостережень, системи автоматичного управління машиною, аналіз рентгенівських знімків у медицині тощо. Сфера цих завдань отримала назву — комп’ютерне бачення (англ. computer vision). Існує багато типових задач комп’ютерного бачення:
        ◦ Класифікація зображення
        ◦ Пошук відомих об’єктів на зображення
        ◦ Сегментація зображення
        ◦ Генерація нових зображень
        ◦ Покращення якості зображення
        ◦ Класифікація відео
        ◦ Локалізація об’єктів на відео в часовому та просторовому вимірах
        ◦ Реконструкція тривимірної сцени

Проблема комп’ютерного зору видається простою, оскільки її банально вирішують люди, навіть зовсім маленькі діти. Тим не менш, це значною мірою залишається невирішеною проблемою, що базується як на обмеженому розумінні біологічного зору, так і через складність сприйняття зору в динамічному і майже нескінченно мінливому фізичному світі. Основними критеріями оцінки завдань є якість роботи, та швидкість (обмеженість обчислювальних ресурсів). Задачі комп’ютерного бачення зазвичай аналізують послідовність зображень (відео), тому важливим критерієм є швидкість задля отримання результату без затримки та уникнення втрати часової інформації.

Провідним напрямком галузі комп'ютерного бачення є використання нейронних мереж. Нейронні мережі - це потужний математичний апарат, що має важливу властивість узагальнення. Завдяки цій властивості нейронні мережі здатні узагальнювати великий набір даних та робити передбачення на даних, що ніколи не зустрічалися раніше. Важливо зауважити, що для навчання нейронних мереж потрібно не лише дані, але й у більшості випадків - анотації, процес отримання яких на сьогодення не може бути автоматизований. Таким чином у області машинного навчання найважливішою та найдорожчою складовою є дані.
 

\textbf{\textit{Актуальність теми.}} Дослідження проблеми малої кількості даних є актуальною проблемою в галузі машинного навчання, оскільки процес отримання проанатованих даних є виключно ручною працею і на сьогодні не може бути автоматизований повністю (лише певні рутині етапи). Не маючи набору даних, немає можливості відворити навіть результати існуючих алгоритмів та моделей. Можливість навчання нейроних мереж на невеликій кількості даних суттєво пришвидшить та здешевить процес навчання моделей. Особливо гостро проблема недостачі даних проявляється у задачі сегментації зображень, де на анотацію одного зображення витрачається декілька хвилин людського часу.

\textbf{\textit{Метою}} даної роботи є розробка архітектури нейронної мережі, яка здатна вирішувати задачу сегментації зображень, тренуючись на малому наборі даних. 

Досягнення зазначеної мети передбачає виконання таких \textbf{\textit{завдань}}:
        \begin{enumerate}[1)]

\item
дослідити математичний апарат та ідею навчання на малому наборі даних
\item
реалізувати архітектуру нейронної мережі та натренувати модель для сегментації довільних об'єктів
\item
порівняти результати з існуючими моделями

\end{enumerate} 

\textbf{\textit{Об'єкт дослідження}} -- нейронні мережі для задачі сегментації зображень

\textbf{\textit{Предмет дослідження }} -- методи нейронних мереж на обмеженому наборі даних 

Робота складається зі вступу, ? роділів та висновків.
В першому розділі 

\newpage
\section{ Підходи роботи з зображенням }
\subsection{ Розпізнавання зображень }
\subsubsection{ Теоретичне підґрунтя роботи із зображенням }

Цифрове зображення — числова репрезентація двовимірного зображення. Зазвичай цифрове зображення дискретне. Дискретність зображення зумовлена обмеженнями технічного забезпечення камери. Бувай винятки, коли колір зображення є неперервним (колір задається довжиною хвилі). Проте, такий тип потрібен лише у вузькоспеціалізованих завданнях, та в повсякденному житті не використовується. В теорії зручно подавати зображення у вигляді матриці

\begin{equation}
    I = \left( 
\begin{matrix}
	i_{1,1} & i_{1,2} & \cdots & l_{1,n} \\
	i_{2,1} & i_{2,2} & \cdots & l_{2,n} \\
	\cdots &  \cdots & \cdots & \cdots \\
	i_{m,1} & i_{m,2} & \cdots & l_{m,n} 
\end{matrix}	 
  \right)
 
\end{equation}

В області $\Omega$ потрібно знайти функцію $u(x,t)$, що задовольняє (в певному сенсі) однорідне хвильове рівняння:
\begin{equation}\label{eq}
\frac{\partial^2u(x,t)}{\partial t^2}-\Delta u(x,t)=0, \,\, (x,t)\in Q,
\end{equation} 

\newpage

\section{Подання розв'язку задачі}
\subsection{Потенціали }

\newpage
\section{Чисельні розв'язки модельних задач}


Розглянемо графіки результатів для $u_0^h$, $u_5^h$ при різних розбиттях. Точка спостереження $x$ змінюється від 1.2 до 4 відповідно. Для розв'язування задачі вибрано параметри $\alpha=2$, $\beta=0$, $\sigma=2$, $\gamma=2$, $b=1$.

\begin{figure}[h!]
	\begin{center}
		\begin{tikzpicture}
		\begin{axis}[
		width=450pt,
		height=250pt,
		xlabel={$x$},
		ylabel={$u_0(x)$},
		legend entries={$Exact$,$M=108$, $M=192$,$M=300$,$M=432$,$M=588$,$M=768$, $M=972$, $M=1200$,$M=1452$}
		%legend entries={$N=10$,$N=20$,$N=30$, $N=50$}
		]
		%\addplot table {data/wave_10_1.dat};
		\addplot table {E0Data/Analytic.dat};
		\addplot table {E0Data/108.dat};
		\addplot table {E0Data/192.dat};
		\addplot table {E0Data/300.dat};
		\addplot table {E0Data/432.dat};
		\addplot table {E0Data/588.dat};
		\addplot table {E0Data/768.dat};
        \addplot table {E0Data/972.dat};
        \addplot table {E0Data/1200.dat};
        \addplot table {E0Data/1452.dat};
		\end{axis}
		\end{tikzpicture}
	\end{center}
	\caption{Графіки $u_0$ для різних розбиттів}
	\label{extDSphereWave1G_graph}
\end{figure}



Розглянемо абсолютні і відносні похибки та порядки збіжності для $u_0^h$, $u_5^h$ та $u_10^h$.
\begin{table}[h!]
{\caption{Похибки та порядки збіжності для $u_0^h$.}
\footnotesize
\centering
%\caption{} % title of Table
\label{u_0_rate_du_dn}
\pgfplotstableread{data/u0.dat}\datatable
\pgfplotstabletypeset[col sep=tab,
sci precision=2, precision=2, sci 10e, zerofill,sci sep align,set thousands separator={},
columns/M/.style={fixed, precision=0, column name=\textsc{$M$}},
columns/rateofconvergence/.style={column name=\textsc{$eoc_0$},precision=2},
columns/err/.style={column name=\textsc{$\epsilon_0^h (\%)$}},
columns/delta/.style={column name=\textsc{$\delta_0^h$},precision=2},
every head row/.style={before row=\toprule,after row=\midrule},
every last row/.style={after row=\bottomrule}
]{\datatable}

}
\end{table}



\section*{\centering ВИСНОВКИ}
\addcontentsline{toc}{section}{Висновки}

У маґістерській роботі розглянуто початково-крайову задачу для однорідного хвильового рівняння з однорідними початковими умовами та імпедансною крайовою умовою. Стосовно неї отримано такі основні результати:

\begin{enumerate}

\item Доведено існування та єдиність ... 


\end{enumerate}

Отримані результати утворюють основу для ....



\newpage

\addcontentsline{toc}{section}{Список використаних джерел}
\begin{thebibliography}{7}

\bibitem{BHD} FSS-1000: A 1000-Class Dataset for Few-Shot Segmentation




\end{thebibliography}
\end{document}


